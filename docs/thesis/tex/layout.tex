\section{Layout}\label{sec:layout}
The present document serves as a template. If you type your thesis in it, your report has the correct format. Below are described the layout requirements.

The reports must be laid out for A4 paper, with horizontal and vertical margins of 25 mm. Page numbers must be on the bottom right-hand side of the page, 10 mm below the text. The title, subtitle, authors and abstract must be laid out as a single column across the whole width of the page, but the main text must be set out in two columns. Columns are 77 mm wide, with a 6 mm space in between.
The font must be Palatino Linotype (Regular), or similar (e.g. Computer Modern by Latex). Times Roman, nor Georgia, the University of Groningen preferred font of the house style, is not recommended because it is not really suitable for long texts. The font size is 10 points for the main text, with single line spacing. Boldface type should only be used for subheadings. Italicized text is permitted. Underlining should not be used – it belongs to the era of manuscripts and typewriters!

Paragraphs should be left and right aligned, and words may be hyphenated. There should be no indent in the first paragraph of a section. Subsequent paragraphs in the section should have a 6 mm first-line indent.

\subsection{The title material}\label{sec:_title_material}
The title of the report is printed in 16-pt bold. The title is fully capitalized, but use 14-pt capitals (or use `small caps') where lower-case letters would normally be used.  Content words in the title must begin with a capital letter (`title caps'). Leave 3 points of empty space under the title.

The title is followed by the name of the name of the author(s), with the student number(s) after the name (remove the student number from the final version that you upload to the repository).
Leave 6 points before the title, and 12 points after the last line. This is followed by the abstract.

All the title details are centred on the page.

\subsection{The abstract}\label{sec:_abstract_layout}
The abstract should ideally consist of one paragraph containing a brief description of what is discussed in the document. Its maximum length is 250 words. Any further paragraphs in the abstract should have a 6 mm first-line indent, as in the main text. The abstract begins with the word Abstract (in bold), followed by a colon. After this comes the main text, beginning with a capital letter.

The font size for the abstract is 9 pt. The abstract is presented as a single column across the whole page, with a 6 mm indent on both sides. The abstract is followed by 6 points of empty space, then the beginning of the main text.

\subsection{The subheadings}\label{sec:_the_subeadings}
It is advisable not to have more than two subheading levels in a short text. All subheadings should be in bold, numbered, with 12 points of empty space before them and 3 points after them. The font size for first-level subheadings is 12 pt. All other levels are 10 pt. Subheadings must be left-aligned.

Arabic numerals should be used for subheadings. Single numbers (1, 2, etc.) are used for the first level, double numbers (1.1, 1.2 etc.) for the second level, three numbers for the third level, and so on. Numbers are followed by a full stop.

Subheadings begin with a capital letter, but contain no further capitals. They do not end with a full stop.

\subsection{The main text}\label{sec:_the_main_text}
The main text was discussed in the first part of this section (10 pt, left and right aligned). It is advisable to check spelling and hyphenation (Latex makes many mistakes with this in Dutch). Make sure there are no line breaks in the middle of formulae or between a number and unit (e.g. 6 cm). We strongly recommend that you use a spelling checker. Also check that tremas and accents, when required, are used correctly.

Italics are used for words from other languages (except for terms that have become commonplace), Latin abbreviations, new terminology/concepts and for emphasis. Short quotes may also be italicized. Quotes must be placed between quotation marks. Make sure that you use opening and closing quotation marks correctly, i.e. `correct' and 'not correct' and ``also correct'' and ''wrong again''.

The short dash - (hyphen) is used to join words or to separate the syllables of a word. The `en dash' is used as a minus sign and to indicate series of numbers ($-1$, $2 - 5$). The `em-dash' -is used as a pause or parenthesis.

Numbers are best presented using modified American notation, i.e. points are used where a comma would be used in Dutch, and vice versa. One and a half is therefore written as 1.5 and not as 1,5. Scientific notation should be used for large numbers: \SI{6.022e-23} and, if necessary, spaces are used to separate large numbers into groups of three digits: $4 \; 294 \; 967 \; 296$. This also works after the decimal point: $3.141 \; 592 \; 653 \; 589 \; 79$.

\subsection{Formulae}\label{sec:_formulae}
Formulae must be written in the current notations used in mathematics, logic and physics.  Make sure that symbols are properly explained. Use the same font as in the main text.

\begin{equation}\label{eq:some_formulae} \delta J =  A \frac{dL}{dq} \delta q B + C \frac{dL}{dq} - \frac{d}{dt}\frac{dL}{dq} D~\delta q ~dt
\end{equation}
This equation can be referred to as equation \ref{eq:some_formulae}, whereby the first number indicates the section and the second number is the number of the equation within the section. 

Formulae must be left-aligned, and their numbers right-aligned.

\subsection{Pseudocode}\label{sec:_pseudocode}
It is best to express algorithms in pseudo code. This is a detailed high-level description using notations from mathematics and logic, and concepts from structured or object-oriented programming.

The font size for pseudo code is 9 pt and, as with figures and tables, should be set apart from the main text. As in mathematical formulae, the names of  variables are italicized. Control words in the code (such as \textbf{if}, \textbf{then}, \textbf{else}, \textbf{for},\textbf{while}) should be in boldface. Function names (such as {\scshape Sort, FindShortest, ProcessSensor}, etc.) are printed in \textit{small caps}. Line numbers must be used if reference is made to individual lines of code. Blocks of code are indented, and conclude with end and the relevant control word, such as end for, end while or end else.

When pseudo code is used, an explanatory caption must be included above the code. The caption should be 9 pt bold and begin with `Algorithm' followed by the section number and the number of the algorithm within the section. The reference to it in the text is `see algorithm \ref{alg:some_algorithm}'.

% see http://en.wikibooks.org/wiki/LaTeX/Algorithms_and_Pseudocode
\begin{algorithm}[!tbp] 
\caption{Calculate $y = x^n$}
\label{alg:some_algorithm}
\begin{algorithmic}
\REQUIRE $n \geq 0 \vee x \neq 0$
\ENSURE $y = x^n$
\STATE $y \Leftarrow 1$
\IF{$n < 0$}
\STATE $X \Leftarrow 1 / x$
\STATE $N \Leftarrow -n$
\ELSE
\STATE $X \Leftarrow x$
\STATE $N \Leftarrow n$
\ENDIF
\WHILE{$N \neq 0$}
\IF{$N$ is even}
\STATE $X \Leftarrow X \times X$
\STATE $N \Leftarrow N / 2$
\ELSE[$N$ is odd]
\STATE $y \Leftarrow y \times X$
\STATE $N \Leftarrow N - 1$
\ENDIF
\ENDWHILE
\end{algorithmic}
\end{algorithm}

\subsection{Figures}\label{sec:_figures}
Figures must be comprehensible and the text they contain must be clearly legible. Bitmaps may only be used in the case of photos. They are not acceptable for line diagrams and graphs, unless the quality and resolution are so high that there is no visible difference.

Graphs must have clear axes with the values marked, an indication of what they show and, where relevant, the units used. If more than one thing is shown in the graph, a legend must be provided or an explanation given in the caption below the graph. The various elements of the graph must be clearly distinguishable, even in black and white! Where relevant, graphs must also include reliability intervals or other information about possible variation.

Figures must be either one or two columns wide, and must be placed either above or below the margins.

An explanatory caption must be included below the figure. The caption begins with the word ‘Figure’, followed by the number of the section, a point and the number of the figure within the section. This is followed by a colon and the rest of the caption, beginning with a capital letter. The caption should be in 9-pt bold. Six points of blank space should be left between the caption of a figure and the text.

The in-text reference is ‘see Figure 4.1’. Figures should be placed as close as possible to the first reference to them.

\begin{figure}[!tbp]
    \centering
        \includegraphics[width=7.7cm]{img/grafiek.png}
    \caption{Graph of the function $sin(x) \cdot sin(y)$. This figure is a bitmap, but of sufficiently high quality to be acceptable. Take note of the legible axes and labels.}
    \label{fig:afb1}
\end{figure}

\subsection{Tables}\label{sec_tables}
Tables can be used if precise values are important, or in cases where a graph would not make things clearer. Tables may also be placed in the middle of the text if they are small enough. Otherwise, as with figures, they should be placed directly above or below the margin.

A table consists of rows and columns. These must be clearly labelled (in bold). Numbers in a table are in ordinary typeface. The font size for all text in a table is 9 pt. Numbers are centred or aligned around the decimal point. In principle, rows and columns are separated by a single half-point line, but a certain amount of creativity may be used to group together related parts of the table.

Tables must be provided with a caption (also 9-pt bold). Convention dictates that the caption should be \textit{placed} above the table. It begins with `Table', followed by the section number, a point, and the number of the table within the section. The in-text reference is `see Table \ref{tab:some_table}'. One blank line is left above and below the table. It is also allowed to use Verdana 8 pt in the table if that is clearer.

\begin{table}[!bp]
\caption{Number of student passes and fails per year.}
\label{tab:some_table}
\begin{tabular}{|l|l|l|l|}
\hline
& 2003 & 2004 & 2005 \\ \hline
geslaagd & 16 & 19 & 17 \\ \hline
gezakt & 20 & 23 & 19 \\ \hline
\end{tabular}
\end{table}

\subsection{References}\label{sec:_references}
It is very important that the references in a document conform to a particular standard, because it must be possible for readers to find the background material referred to. The author, year, title of the book/journal, publisher and issue number must all be provided. This report should follow the conventions of the American Psychological Association. These can be found on the internet, but it is helpful to look at a number of examples. Other conventions may be used but this has to be discussed and approved of by your supervisor beforehand.

Articles are referred to with the name(s) of the author(s) and the year \citep{dawkins76}. If the source has two authors, the names are separated by an ampersand \citep{berrah99} but if there are three or more authors, ‘et al.’ is used\citep{schwartz97}. A reference may include several sources \citep{cooper52, Crothers78} or even more than one paper by the same author \citep{Kirby98, Kirby99, Kirby00, Oliphant93, Oliphant96}. Note the use of commas and semi-colons.
It is important to avoid references to vague sources as far as possible. It is best to avoid sources that are \textit{in press, unpublished manuscripts or online resources.}