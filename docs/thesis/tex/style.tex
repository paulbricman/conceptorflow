\section{Style}\label{sec:style}
Academics and scientists have a product to sell: the results of their research and the hypotheses that those results support. The readers to whom the product is offered are extremely critical and knowledgeable on the subject concerned. Academic and scientific texts must therefore be written with this in mind. In everything they write, writers must ask themselves `Does this help to convey my message to my envisaged audience?'. In addition, the audience/readers are used to seeing the message presented in a particular style. It is important not to deviate too much from this style, because this would generate a `resistance' to the actual style used and thus, inevitably, to the content.

\subsection{The academic style}\label{sec:_academic_style}
The academic writing style is different to the style used in newspapers, novels and personal correspondence. The style used in `serious' scholarly articles is even different to the style used in popular-science articles and books.

The most notable thing about the academic style is that it generally avoids the use of the first and second person pronouns (I, you, we). Texts are often written using impersonal constructions or the passive voice. This is perhaps explained by the fact that, in the ideal academic process, it is not a matter of the work that the writer has carried out personally, but that each academic/scientific article is, in principle, a non-personal contribution to the advancement of knowledge.

It may be that the use of `we' is an exception. In English this style is known as the `tutorial we' style. Some journals recommend this style (the `Here we show...' found in almost every article in the journal Nature is a good example) while other journals expressly advise against it. Particularly in a thesis by a single author, the use of `we' is comical rather than functional. In your thesis you may use ``I'', but often it is better to avoid using it (e.g., by writing ``this study intended to test …'' rather than ``I wanted to see …'')

Style conventions do not remain constant. They change over time – as is clearly seen in older articles – and may also vary considerably from discipline to discipline. One thing that remains constant is the fact that the style must be geared to readability. Overblown formal language is of no use to anyone. It is therefore always a good idea to ask someone to read through your texts in order to assess their readability (and content, of course).

\subsection{Tips for a good writing style}\label{sec:_tips_for_writing}
The most useful tip is to \textit{read many academic articles} to see what works and what does not, and what is customary and what is unusual. It also helps to read articles written over a wide range of years. Certain conventions regarding form and content change over the years, and are arbitrary to a certain extent, but a good article will always be a good article. Likewise, a bad article will always be a bad article, and it can certainly do no harm to look at a number of articles in this category in order to gain a better impression of why certain things do not work. 

The second tip is to read a book about how to write. Useful resources on the subject of writing academic and scientific reports are available on the internet. A less obvious source, but certainly a very useful one, are the guidelines for Wikipedia authors.

Another useful tip for good writing – at paragraph level at least – is to build up the paragraphs in a fairly consistent way. One paragraph deals with one subject, or links two subjects. The first sentence indicates what the paragraph will be about. The final sentence of the paragraph establishes the link to the next paragraph. The detailed writing comes in between these two sentences. Obviously it is not necessary to structure every paragraph in this way (texts would easily become illegible if they were all written strictly according to the rules), but this is a useful guide.

Another way to improve your text is to consider the function of everything you write in relation to the whole. If you cannot give a positive answer to the question `Is this information relevant to what I am trying to say?', it is better to leave it out. If you cannot give a positive answer to the question `Will my readers understand this?', you know that you will have to explain it in more detail. An almost direct consequence of these criteria is that vague generalizations (`The interest in X is rapidly increasing' or `Many scientists believe that ...') must be avoided, or substantiated – for example with references.

Because writers can be their own least critical readers, it is useful to ask someone else to read your texts, and to take that person's comments seriously. Given that we tend to be lenient with ourselves in self-criticism, if we think something is bad it is almost certain to be bad.